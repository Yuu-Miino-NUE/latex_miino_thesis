\documentclass[a4paper,dvipdfmx,11pt,twoside]{jsreport}

\usepackage{miino_thesis}
\bibliographystyle{junsrt}
\usepackage{cleveref}
\crefname{figure}{図}{図}
\crefname{table}{表}{表}

\doctype{2023年度 卒業論文}
\title{卒業論文のタイトル}
\supervisor{
    指導教員 & 美井野 優
}
\organization{◯◯大学 ◯◯学部\\ ◯◯コース}
\author{学籍番号: 20xxxxxxxx\\山田 太郎}

\begin{document}
\begin{abstract}
    要旨
\end{abstract}

% \frontcover % <-- 表紙を印刷する場合
% \frontcover[\seihon] % <-- 表紙に「正本」と印刷する場合
% \frontcover[\fukuhon] % <-- 表紙に「副本」と印刷する場合

\maketitle

\tableofcontents

\chapter{序論}
本章では,本研究の背景と目的,および本論文の構成について述べる.
\section{背景}
論文には体裁がある.
本 LaTeX スタイルは GitHub 上のレポジトリ\cite{miino2023github}で公開・更新している.
\clearpage
\section{目的}
\clearpage
\section{本論文の構成}

\chapter{理論}

\chapter{実験・結果・考察}

\chapter{結論}

\chapter*{謝辞}

% 参考文献 (BiBTeX)
\bibliography{thesis}

\end{document}